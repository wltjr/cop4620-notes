\documentclass{article}

    \usepackage[tmargin=1in,lmargin=1in,rmargin=1in,bmargin=1in]{geometry}
    \usepackage[T1]{fontenc}
	\usepackage{titlesec}
    \usepackage{makecell}
    \usepackage{multicol}
    \usepackage{longtable}
    \usepackage{soul}
    
    \renewcommand{\familydefault}{\sfdefault}        	 
    \renewcommand\theadfont{\bfseries\sffamily}
	\titleformat{\section}
	{\normalfont\bfseries}
	{\thesection.}{0.5em}{}
\begin{document}

\section*{Problem Set 2: Context-free Grammars, Parsers}
\begin{enumerate}
  \setlength\itemsep{-.25em}
  \item For the following sub-problems, consider the following context-free grammar:
  \begin{enumerate}
    \setlength\itemsep{-.25em}
    \renewcommand{\labelenumii}{\arabic{enumii}.}
    \item S $\rightarrow$ AA\$
    \item A $\rightarrow$ xA
    \item A $\rightarrow$ B
    \item B $\rightarrow$ yB
    \item B $\rightarrow$ $\lambda$
  \end{enumerate}
  \begin{enumerate}
    \setlength\itemsep{-.25em}
    \item What are the terminals and non-terminals of this language?\newline\newline
    \textbf{terminals:} \{x, y, \$\} \newline
    \textbf{non-terminals:} \{S, A, B\} \newline

    \item Describe the strings are generated by this language. Is this a regular language\newline
    	(i.e., could you write a regular expression that generates this language)?\newline\newline
    Yes, this is a regular language. \textbf{re:} \textit{x*y*x*y*}\newline
    	
    \item Show the derivation of the string \textit{xyxxy\$} starting from \textit{S} (specify which productions you used at each step), and give the parse tree according to that derivation.
\begin{multicols}{2}
  \begin{itemize}
    \setlength\itemsep{-.25em}
    \item[] S $\rightarrow$ AA\$
    \item[] \hspace{.9em}$\rightarrow$ xAA\$
    \item[] \hspace{.9em}$\rightarrow$ xBA\$
    \item[] \hspace{.9em}$\rightarrow$ xyBA\$
    \item[] \hspace{.9em}$\rightarrow$ xyA\$
    \item[] \hspace{.9em}$\rightarrow$ xyxA\$
    \item[] \hspace{.9em}$\rightarrow$ xyxxA\$
    \item[] \hspace{.9em}$\rightarrow$ xyxxB\$
    \item[] \hspace{.9em}$\rightarrow$ xyxxyB\$
    \item[] \hspace{.9em}$\rightarrow$ xyxxy\$
  \end{itemize}
  \begin{itemize}
    \setlength\itemsep{-.25em}
    \item[] Rule 1
    \item[] Rule 2
    \item[] Rule 3
    \item[] Rule 4
    \item[] Rule 5
    \item[] Rule 2
    \item[] Rule 2
    \item[] Rule 3
    \item[] Rule 4
    \item[] Rule 5
  \end{itemize}
\end{multicols}
    
    \item Give the first and follow sets for each of the non-terminals of the grammar.\newline
    
\textbf{First sets}\newline
First (B) = First(yB) $\cup$ First($\lambda$) = \{y, $\lambda$\} $\leftarrow$ \textbf{Final set of B}\newline
\indent\hspace{.75cm}First (yB) = First of \{y\} - $\lambda$ = \{y\} - $\lambda$ = \{y\}\newline
\indent\hspace{.75cm}First ($\lambda$) = First of \{$\lambda$\} - $\lambda$ = \{$\lambda$\}\newline
First (A) = First(xA) $\cup$  First(B) = \{x\} $\cup$ \{ y, $\lambda$\} = \{x, y, $\lambda$\} $\leftarrow$ \textbf{Final set of A}\newline
\indent\hspace{.75cm}First (xA) = First of \{x\} - $\lambda$ = \{x\} - $\lambda$ = \{x\}\newline
First (S) = First(AA\$) = (First(A) - $\lambda$) $\cup$ (First(\$) - $\lambda$) = \{x, y, \$\} $\leftarrow$ \textbf{Final set}\newline

\textbf{Follow sets}\newline
Follow (S) = \{\} $\leftarrow$ \textbf{Final set of S}\newline
Follow (A) = First(A\$) $\cup$ Follow(\$) $\cup$ Follow (B)\newline
\indent\hspace{1.7cm}= (First(A) - $\lambda$) $\cup$ \{\$\} $\cup$ Follow (B)\newline
\indent\hspace{1.7cm}= \{x\} $\cup$ \{\$\} $\cup$ Follow (B)\newline
\indent\hspace{1.7cm}= \{x\} $\cup$ \{\$\} $\cup$ \{x, y, \$\} = \{x, y, \$\} $\leftarrow$ \textbf{Final set of A}\newline
Follow (B) = First (B) $\cup$ Follow(A) $\cup$ \st{Follow (B)}\newline
\indent\hspace{1.7cm}= (First(B) - $\lambda$) $\cup$ Follow(A) \newline
\indent\hspace{1.7cm}= \{y\} $\cup$ \{x, \$\} = \{x, y, \$\} $\leftarrow$ \textbf{Final set of B}\newline
    \item What are the predict sets for each production?
    \item Give the parse table for the grammar. Is this an LL(1) grammar? Why or why not?
  \end{enumerate}
  
\end{enumerate}

\end{document}